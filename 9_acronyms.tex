%\usepackage[acronyms,nomain,toc,automake]{glossaries} %this is the proper preamble command
%\makeglossaries also required in preamble

\label{Acronyms}

%============== Your acronyms here! ================%
% \newacronym{reference}{Displayed}{Long description}

\newacronym{ufc}{UFC}{Up Front Control}
\newacronym{acm}{ACM}{Air Combat Manuevering}
\newacronym{lol}{LOL}{laugh out loud}
\newacronym{as}{A/S}{Air to Surface}

% by default this will only print acronyms used in the document, this command will display all acronyms added in this file
\glsaddall[types=\acronymtype]

\newglossarystyle{mylistdotted}{\glossarystyle{listdotted}%
   \renewenvironment{theglossary}{\begin{compactdesc}}{\end{compactdesc}}%
   \renewcommand*{\glsgroupskip}{}}

\begin{singlespacing}
\printglossary[title={Acronyms and Abbreviations},type=\acronymtype, style=mylistdotted]
\end{singlespacing}


%================== How to use this ======================================
%\gls command will display "<full> (<abbrv>)" of first use. On subsequent uses only the abbreviation will be displayed.

%To reset the first use of an acronym, use the command:

%\glsreset{<label>} or, if you want to reset the use status of all acronyms: \glsresetall 

%Similarly, to unset the first use of an acronym so that only the abbreviation will be displayed, use: %\glsunset{<label>} or, for all acronyms:  \glsunsetall

%\acrlong{ }  %Displays the phrase which the acronyms stands for. Put the label of the acronym inside the braces. In the example, \acrlong{gcd} prints Greatest Common Divisor.

%\acrshort{ }  %Prints the acronym whose label is passed as parameter. For instance, \acrshort{gcd} renders as GCD.

%\acrfull{ }  %Prints both, the acronym and its definition. In the example the output of \acrfull{lcm} is Least Common Multiple (LCM).